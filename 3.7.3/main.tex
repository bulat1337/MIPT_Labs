\documentclass[a4paper, 12pt]{article}%тип документа

%отступы
\usepackage[left=2cm,right=2cm,top=2cm,bottom=3cm,bindingoffset=0cm]{geometry}

%Русский язык
\usepackage[T2A]{fontenc} %кодировка
\usepackage[utf8]{inputenc} %кодировка исходного кода
\usepackage[english,russian]{babel} %локализация и переносы

%Вставка картинок
\usepackage{wrapfig}
\usepackage{graphicx}
\graphicspath{{pictures/}}
\DeclareGraphicsExtensions{.pdf,.png,.jpg}
\usepackage{caption}
\usepackage{float}




%оглавление
\usepackage{titlesec}
\titlespacing{\chapter}{0pt}{-30pt}{12pt}
\titlespacing{\section}{\parindent}{5mm}{5mm}
\titlespacing{\section}{\parindent}{5mm}{5mm}
\usepackage{setspace}

%Графики
\usepackage{multirow}
\usepackage{pgfplots}
\pgfplotsset{compat=1.9}

%Математика
\usepackage{amsmath, amsfonts, amssymb, amsthm, mathtools}

%Заголовок
\author{Мотыгуллин Булат}

\title{\textbf{Работа 3.7.3\\
Длинные Линии}}

\date{}

\begin{document}

\maketitle


\section*{Цель работы}
Ознакомится и проверить на практике теорию распространения
электрических сигналов вдоль длинной линии; измерить амплитудо- и фазово-частотные
характеристики коаксиальной линии; определить погонные характеристики такой
линии; на примере модели длинной линии изучить вопрос распределения амплитуды
колебаний сигнала по длине линии.

\section*{Эксперементальная установка}
\begin{wrapfigure}{l}{0.5\textwidth}
  \begin{center}
    \includegraphics[width = 0.5\textwidth]{setup.png}
  \end{center}
  \textbf{\caption{Схема установки.}}
\end{wrapfigure}

Oодин конец коаксиального кабеля с
помощью "тройника" подключен к выходу
"1" генератора и ко входу CH1 осциллографа
(для соединения тройника с генератором
использован короткий коаксиальный кабель;
тройник напрямую подключен к
осциллографу, чтобы сдвиг фазы сигнала на
входе в осциллограф и в начале "длинной
линии" был минимален);

Второй конец кабеля "длинной
линии" подключен ко входу CH2
осциллографа.

\section*{Формулы}

\begin{align*}
    V_{\text{ф}} &= \frac{f_0}{n} l & \quad L_xC_x &= \frac{c^2}{{V_{\text{ф}}}^2} \\
    L_x &= cR_0\sqrt{L_xC_x} & \quad C_x &= \frac{\sqrt{L_xC_x}}{cR_0} \\
   \varepsilon &= 2Cln\left(\frac{r_2}{r_1}\right) & \quad \mu &= \frac{L}{2ln\left(\frac{r_2}{r_1}\right)} \\
    \sigma &= \left( \frac{2C_xV_{\text{ф}}}{c \cdot d \cdot \frac{\alpha}{\sqrt{f}}} \right)^2 & \quad ... &= ...
\end{align*}



\section*{Результаты}

\begin{figure}[H]
    \captionsetup{position=above, skip=2pt}
    \centering
    \caption{Зависимость резонансной частоты от n.}
    \includegraphics[width=0.9\textwidth]{f-n(sin_norm).png}
\end{figure}

Погонные характеристики кабеля:

$$ C_x = (1,00 \pm 0,01) \text{ед. СГС} $$

$$ L_x = (2,25 \pm 0,02) \text{ед. СГС} $$

Диэлектрическая проницаемость:

$$ \varepsilon = 2,25 \pm 0,29 $$

Магнитная восприимчивость:

$$ \mu = 0,99 \pm 0,13 $$


\begin{figure}[H]
    \captionsetup{position=above, skip=2pt}
    \centering
    \caption{Зависимо
сть декремента затухания частоты от от корня частоты гармонического сигнала.}
    \includegraphics[width=0.9\textwidth]{alpha-sqrt(f).png}
\end{figure}

Удельная проводимость провода:

$$ \sigma = (6,34 \pm 0,23) \text{ед.СГС} $$

\section*{Вывод}

Фазовые скорости в условиях согласованной нагрузки и при отсутствии нагрузки совпадают, как и результаты для синусоидального и прямоугольного сигналов. Характеристики кабеля были получены с высокой точностью. График зависимости декремента затухания от квадратного корня частоты показал искажения на малых и больших частотах, что является допустимым для измерений вблизи граничных значений частот.


\end{document}